\chapter{\waveplot}

The \waveplot{} program is a tool for the visualisation of molecular orbitals.
Based on the files created by a calculation performed by \dftbp{} it is capable
of producing three dimensional information about the charge distribution. The
information is stored as cube files, which can be visualised with many common
graphical tools (e.g. \textsc{vmd} or \textsc{Jmol}).

The user controls \waveplot{} through an input file, choosing which
orbitals and charge distributions should be plotted for which spatial
region. Since the information about the shape of the basis functions
is usually not contained in the Slater-Koster files, the coefficients
and exponents for the Slater type orbitals must be entered by the user
as part of the input file.

The \waveplot{} tool offers the following plotting capabilities:
\begin{itemize}
\item Total charge density.
\item Total spin polarisation.
\item Difference between the total charge density and the density
  obtained by the superposition of the neutral atomic densities
  (visualisation of the charge shift).
\item Electron density for individual levels.
\item Real and imaginary part of the wavefunctions for individual levels.
\end{itemize}


\section{Input for \waveplot}

The input file for \waveplot{} must be named \verb|waveplot_in.hsd| and should
be a Human-friendly Structured Data (HSD) formatted file (see Appendix
\ref{sec:hsd}) and must be present in the working directory.

The table below contains the list of the properties, which must occur in the
input file \verb|waveplot_in.hsd|:

\begin{ptableh}
  \kw{Options} & p &   & - &\pref{sec:waveplot.Options} \\
  \kw{DetailedXML} &s &  & - &  \\
  \kw{EigenvecBin} &s &  & - &   \\
  \kw{GroundState} &s &  & Yes &   \\
  \kw{Basis} & p & & - &  \pref{sec:waveplot.Basis} \\
\end{ptableh}
\begin{description}
\item[\is{Options}] Contains the options for \waveplot{}. See
  p.~\pref{sec:waveplot.Options}.
\item[\is{DetailedXML}] Specifies the name of the file, which contains the
  detailed XML output of the \dftbp{} calculation (presumably
  \is{detailed.xml}).
\item[\is{EigenvecBin}] Specifies the name of the file, which contains the
  eigenvectors in binary format (presumably \is{eigenvec.bin}).
\item[\is{GroundState}] Read ground or excited state occupation data from the
  detailed XML output.
\item[\is{Basis}] Contains the definition of the Slater-type orbitals which were
  used as basis in the \dftbp{} calculation. At the moment, due to technical
  reasons this information has to be entered by the user per hand. In a later
  stage, it will be presumably read in by \waveplot{} automatically.
  See p.~\pref{sec:waveplot.Basis}.
\end{description}

Additionally optional definitions may also be present:
\begin{ptableh}
  \kw{ParserOptions} & p & & \cb & \pref{sec:waveplot.ParserOptions} \\
\end{ptableh}

\subsection{Options}
\label{sec:waveplot.Options}

This property contains the options (as a list of properties), which
the user can set, in order to influence the behaviour of \waveplot{}.
The following properties can be specified:

\begin{ptable}
  \kw{PlottedRegion} &p|m&  & - &\pref{sec:waveplot.PlottedRegion}\\
  \kw{NrOfPoints} & 3i & & - &  \\
  \kw{PlottedKPoints} & i+|m & \textrm{periodic system} & - & \\
  \kw{PlottedLevels} & i+|m &  & - & \\
  \kw{PlottedSpins} & i+|m & & - &  \\
  \kw{TotalChargeDensity} & l & & No &  \\
  \kw{TotalSpinPolarisation} & l&  & No & \\
  \kw{TotalChargeDifference} & l& & No & \\
  \kw{TotalAtomicDensity} & l & & No &  \\
  \kw{ChargeDensity} & l & & No &  \\
  \kw{RealComponent} & l & & No &  \\
  \kw{ImagComponent} & l & \textrm{complex wavefunction} & No & \\
  \kw{FoldAtomsToUnitCell} & l & \textrm{periodic system} & No & \\
  \kw{FillBoxWithAtoms} & l & & No & \\
  \kw{NrOfCachedGrids} & i & & -1 &  \\
  \kw{Verbose} & l & & No &  \\
  \kw{RepeatBox} & 3i & & \{1 1 1\} & \\
  \kw{ShiftGrid} & l & & Yes & \\
  \hline
\end{ptable}
\begin{description}
\item[\is{PlottedRegion}] Regulates the region which should be
  plotted. See p.~\pref{sec:waveplot.PlottedRegion}.
\item[\is{NrOfPoints}] Specifies the resolution of the equidistant grid on
  which the various quantities should be calculated. The three
  integers represent the number of points along the three vectors of
  the parallelepiped specifying the plotted region. The number of all
  calculated grid points is the product of the three integers.

Example:
\begin{verbatim}
NrOfPoints = { 50 50 50 }     # 125 000 grid points
\end{verbatim}

\item[\is{PlottedKPoints}] The list of integers specified here represent the k-points, at which the
  molecular orbitals should be plotted. The first k-point in the original \dftbp{} calculation is
  represented by "1". The order of the specified k-points does not matter. (You can also use
  index specification expressions as described in appendix \ref{sec:index_selection}.) The actual
  list of molecular orbitals to plot is obtained by intersecting the specifications for
  \is{PlottedKPoints}, \is{PlottedLevels} and \is{PlottedSpins}.  The option is ignored if the
  original calculation was not periodic.

Example:
\begin{verbatim}
PlottedKPoints =  1 3 5      # The 1st, 3rd and 5th k-point is plotted
\end{verbatim}

\item[\is{PlottedLevels}] The list of integers specified here represent the states which should be
  plotted. The first (lowest lying) state in the original \dftbp{} calculation is represented by
  "1". The order of the specified states does not matter. (You can also use  index specification
  expressions as described in appendix \ref{sec:index_selection}.)  The actual list of molecular
  orbitals to plot is obtained by intersecting the specifications for \is{PlottedKPoints},
  \is{PlottedLevels} and \is{PlottedSpins}.

Example:
\begin{verbatim}
PlottedLevels = 1:-1     # All levels plotted
\end{verbatim}

\item[\is{PlottedSpins}] The list of integers specified here represent the spins, for which the
  molecular orbitals should be plotted. The first spin in the original \dftbp{} calculation is
  represented by "1". The order of the specified spins does not matter. (You can also use index
  specification expressions as described in appendix \ref{sec:index_selection}.) The actual list of
  molecular orbitals to plot is obtained by intersecting the specifications for \is{PlottedKPoints},
  \is{PlottedLevels} and \is{PlottedSpins}.

Example:
\begin{verbatim}
PlottedSpins =  1 2      # Both spin-up and spin-down plotted
\end{verbatim}

\item[\is{ChargeDensity}] If true, the absolute square of the
  wavefunction is plotted for the selected molecular orbitals.

\item[\is{RealComponent}] If true, the real component of the
  wavefunction is plotted for the selected molecular orbitals.

\item[\is{ImagComponent}] If true, the imaginary component of the
  wavefunction is plotted for the selected molecular orbitals. This
  option is only parsed, if the wavefunctions in the \dftbp{}
  calculation were complex.

\item[\is{TotalChargeDensity}] If true, the total charge density of
  the system is plotted.

\item[\is{TotalSpinPolarisation}] If true, the total spin polarisation of
  the system (difference of the spin up and spin down densities) is
  plotted. This option is only interpreted if the processed \dftbp{}
  calculation was spin polarised.

\item[\is{TotalChargeDifference}] If true, the difference between the
  total charge density and the charge density obtained by superposing
  the neutral atomic densities is plotted.

\item[\is{TotalAtomicDensity}] If true, the superposed neutral atomic
  densities are plotted.

\item[\is{FoldAtomsToUnitCell}] If true, the atoms are folded into the
  parallelepiped unit cell of the crystal.

\item[\is{FillBoxWithAtoms}] If true, the geometry is extended by those
  periodic images of the original atoms, which falls in the plotted
  region or on its borders.  It sets \is{FoldAtomsToUnitCell} to Yes.

\item[\is{NrOfCachedGrids}] Specifies how many grids should be cached
  at the same time. The value \is{-1} stands for as many as necessary
  to be as fast as possible. Since the plotted grids could eventually
  become quite big, you should set it to some positive non-zero value
  if you experience memory problems.

Example:
\begin{verbatim}
NrOfCachedGrids = 5     # Maximal 5 cached grids
\end{verbatim}

\item[\is{RepeatBox}] The three integers specify how often the plotted
  region should be repeated in the generated cube files. Since
  repeating the grid is not connected with any extra calculations,
  this is a cheap way to visualise a big portion of a solid.  You want
  probably set the \is{FillBoxWithAtoms} option to Yes to have the
  atoms also repeated (otherwise only the plotted function is
  repeated).  In order to obtain the correct picture, you should set
  the plotted region to be an integer multiple of the unit cell of the
  crystal. Please note, that the phase of the wavefunctions in the
  repeated cells will be incorrect, except in the $\Gamma$-point.

Example:
\begin{verbatim}
  RepeatBox = { 2 2 2 }     # Visualising a 2x2x2 supercell
\end{verbatim}

\item[\is{ShiftGrid}] Whether the grid should be shifted, so that the specified
  origin lies in the middle of a cell and the grid fills out the specified
  plotted region symmetrically. The default is \is{Yes}. If set to \is{No}, the
  specified grid origin will be at the edge of a cell.

\item[\is{Verbose}] If true, some extra messages are printed out
  during the calculation.

\end{description}


\subsubsection{PlottedRegion}
\label{sec:waveplot.PlottedRegion}

Specifies the region, which should be included in the plot. You can
specify it explicitly (as property list), or let \waveplot{} specify
it automatically using either the \islcb{UnitCell}{sec:waveplot.UnitCell} or the
\islcb{OptimalCuboid}{sec:waveplot.OptimalCuboid} methods.

\paragraph{Explicit specification}
Specifies origin and box size explicitly.

\begin{ptable}
  \kw{Origin} & 3r & & - & \\
  \kw{Box} & 9r & & - &  \\
\end{ptable}

\begin{description}
\item[\is{Origin}]\modif{\modtype{length}} Specifies the xyz
  coordinates of the origin as three real values.

\item[\is{Box}]\modif{\modtype{length}} Specifies the three vectors
  which span the parallelepiped of the plotted region. The vectors are
  specified sequentially ($a_{1x}$ $a_{1y}$ $a_{1z}$ $a_{2x}$ $a_{2y}$
  $a_{2z}$ $a_{3x}$ $a_{3y}$ $a_{3z}$). You are allowed to specify an
  arbitrary parallelepiped with nonzero volume here. Please note,
  however, that some visualisation tools only handles cube files with
  cuboid boxes correctly.
\end{description}

Example:
\begin{verbatim}
  PlottedRegion = {
    Origin = { 0.0 0.0 0.0 }
    Box [Angstrom] = {
      12.5   12.5  -12.5
      12.5  -12.5   12.5
     -12.5   12.5   12.5
    }
  }
\end{verbatim}


\paragraph{UnitCell\{\}}
\label{sec:waveplot.UnitCell}

For the periodic geometries, this method specifies the plotted region to be
spanned by the three lattice vectors of the crystal. The origin is set to (0 0
0). For cluster geometries, the smallest cuboid containing all atoms is
constructed. For a cluster geometry the \iscb{UnitCell} object may have the
following property:

\begin{ptable}
  \kw{MinEdgeLength} & r & & 1.0 &  \\
\end{ptable}
\begin{description}
\item[\is{MinEdgeLength}]\modif{\modtype{length}} Minimal side length of
  the cuboid, representing the plotted region.  This helps to avoid
  cuboids with vanishing edge lengths (as it would be the case for a
  linear molecule).
\end{description}

Example:
\begin{verbatim}
  PlottedRegion = UnitCell {
    MinEdgeLength [Bohr] = 2.0
  }
\end{verbatim}


\paragraph{OptimalCuboid\{\}}
\label{sec:waveplot.OptimalCuboid}

Specifies the plotted region as a cuboid, which contains all the atoms and
enough space around them, that no wavefunctions are leaking out of the cuboid.
This object does not have any parameters.

Example:
\begin{verbatim}
  PlottedRegion = OptimalCuboid {}
\end{verbatim}


\subsection{Basis}
\label{sec:waveplot.Basis}

The basis definition is done by specifying the following properties:

\begin{ptable}
  \kw{Resolution} & r &  & - &  \\
  \textit{ElementName1} & p & & - & \pref{sec:waveplot.speciesBasis} \\
  \textit{ElementName2} & p & & - &  \pref{sec:waveplot.speciesBasis} \\
  \hspace*{0.8cm}$\vdots$ & & & & \\
\end{ptable}
\begin{description}
\item[\is{Resolution}] Specifies the grid distance used for
  discretising the radial wavefunctions.  Setting it too small, causes
  a long initialisation time for \waveplot{}. Setting it too high
  causes a very coarse grid with bad mapping and inaccurate charges.
  Values around $0.01$ seem to work fine. (Units must be in Bohr.)

\item[\textit{ElementName1}] Basis for the first atom type. The name
  of this property is the name of that atom type.
\item[\textit{ElementName2}] Basis for the second atom type. The name of this
  property is the name of that atom type.
\end{description}

Before describing the properties (and their sub-properties) in detail, the full
basis definition for carbon (sp) and hydrogen (s) should be presented as
example:


\begin{verbatim}
Basis = {
  Resolution = 0.01
  C = {                          # Basis of the C atom
    AtomicNumber = 6
    Orbital = {                 # 2s orbital
      AngularMomentum = 0
      Occupation = 2
      Cutoff = 4.9
      Exponents = {  6.00000     3.00000     1.50000 }
      Coefficients = {
         1.050334389886e+01   2.215522018905e+01   9.629635264776e+00
        -4.827678012828e+01  -5.196013014531e+00  -2.748085126682e+01
         3.072783267234e+01  -1.007000163584e+01   8.295975876552e-01
      }
    }
    Orbital = {                 # 2p orbital
      AngularMomentum = 1
      Occupation = 2
      Cutoff = 5.0
      Exponents = {  6.00000     3.00000     1.50000  }
      Coefficients = {
        -2.659093036065e+00  -6.650979229497e+00  -1.092292307510e+01
         2.190230021657e+00  -9.376762008640e+00  -5.865448605778e-01
         8.208019468802e+00  -2.735743196612e+00   2.279582669709e-01
      }
    }
  }
  H = {                           # Basis for the H atom
    AtomicNumber = 1
    Orbital = {                  # 1s orbital
      AngularMomentum = 0
      Occupation = 1
      Cutoff = 4.2
      Exponents = {  2.00000     1.00000 }
      Coefficients = {
         1.374518455977e+01   1.151435212242e+01   2.072671588012e+00
        -1.059020844305e+01   3.160957468828e+00  -2.382382105798e-01
      }
    }
  }
}
\end{verbatim}




\subsubsection{Basis for an atom type}
\label{sec:waveplot.speciesBasis}

The actual basis for every atom type is specified as a property with
the name of that type:

\begin{ptable}
  \kw{AtomicNumber} &  i&  & - &  \\
  \kw{Orbital} & p &  & - & \pref{sec:waveplot.Orbital} \\
  \hspace*{0.8cm}$\vdots$ & & & & \\
\end{ptable}
\begin{description}
\item[\is{AtomicNumber}] The atomic number of the species. This is not
  needed in the actual calculations, but for creating proper
  cube-files.
\item[\is{Orbital}] Contains the parameters of the orbitals. For every
  orbital a separate \is{Orbital} block must be created. See below.
\end{description}


\paragraph{Orbital}
\label{sec:waveplot.Orbital}

For every orbital there is an orbital block which specifies the radial
wavefunction. Thereby the following properties must be used:
\begin{ptable}
  \kw{AngularMomentum} & i & & - & \\
  \kw{Occupation} & r & & - &  \\
  \kw{Cutoff} & r& & - & \\
  \kw{Exponents} & r+ &  & -  & \\
  \kw{Coefficients} & r+ &  & - &\\
\end{ptable}

\begin{description}
\item[\is{AngularMomentum}] Angular momentum of the current orbital. ($s$ -- 0,
  $p$ -- 1, $d$ -- 2, $f$ -- 3)

\item[\is{Occupation}] Occupation of the orbital in the neutral ground state.
  (Needed to obtain the right superposed atomic densities.)

\item[\kw{Cutoff}] Cutoff for the wave function. You should choose a value,
  where the value of $4\pi r^2 \left|R(r)\right|^2$ drops below $10^{-4}$ or
  $10^{-5}$.  $R(r)$ is the radial part of the wave function. If you do not have
  the possibility to visualise the radial wave function, take the half of the
  longest distance, for which an overlap interaction exists in the appropriate
  homonuclear Slater-Koster file. (Value must be entered in Bohr.)

\item[\is{Exponents}] The radial wave function with angular momentum $l$ has the
  form:
  \begin{equation}
    \label{eq:raddef}
    R_l(r) = \sum_{i=1}^{n_{\text{exp}}}\,
      \sum_{j=1}^{n_{\text{pow}}} c_{ij}\, r^{l+j-1} e^{-\alpha_i r}
  \end{equation}
  This property defines the multiplication factors in the exponent ($\alpha_i$).

\item[\is{Coefficients}] This property contains the coefficients $c_{ij}$ as
  defined in equation \eqref{eq:raddef}. The sequence of the coefficients must
  be as follows:

  $c_{11}$ $c_{12}$ \dots\ $c_{1n_{\text{pow}}}$ $c_{21}$ $c_{22}$
  \dots\ $c_{2n_{\text{pow}}}$ \dots


\end{description}


\subsection{ParserOptions}
\label{sec:waveplot.ParserOptions}

This block contains the options, which are effecting only the behaviour of the
HSD parser and are not passed to the main program.
\begin{ptable}
  \kw{IgnoreUnprocessedNodes} &l & & No & \\
  \kw{StopAfterParsing} &l & & No & \\
\end{ptable}
\begin{description}

\item[\is{IgnoreUnprocessedNodes}] By default the code stops if it
  detects unused or erroneous keywords in the input. This {\em
    dangerous} flag suspends these checks. Use only for debugging
  purposes.

\item[\is{StopAfterParsing}] If set to \is{Yes}, the parser stops
  after processing the input and written out the processed input to
  the disc. It can be used to make sanity checks on the input without
  starting an actual calculation.

\end{description}
