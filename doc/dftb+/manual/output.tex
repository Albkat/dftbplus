\chapter{Output of \dftbp}
\label{sec:output}

This chapter contains the description of some of the output files of
\dftbp{} where the output format is not self documenting. Unless
indicated otherwise, numbers in the output files are given in atomic
units (with Hartree as the energy unit).

\section{hamsqrN.dat, oversqr.dat}
\label{sec:hamsqr}
\index{hamsqr.dat}\index{oversqr.dat}

The files \verb|hamsqrN.dat| and \verb|oversqr.dat| contain the square
(folded) Hamiltonian and overlap matrices. The number \verb|N| in the
filename \verb|hamrealN.dat| indicates the spin channel. For spin
unpolarised calculation it is 1, for spin polarised calculation it is
1 and 2 for spin-up and spin-down, respectively while for non-colinear
spin it is charge, $x$, $y$ and $z$ for 1, 2, 3 and 4. Spin orbit is
not currently supported for this option.

Only non-comment lines (lines not starting with "\#") are documented:
\begin{itemize}

\item Flag for signalling if matrix is real (\verb|REAL|), number of
  orbitals in the system (\verb|NALLORB|), number of kpoints
  (\verb|NKPOINT|). For non-periodic (cluster) calculations, the
  number of kpoints is set to 1.

\item For every $k$-point:
  \begin{itemize}
  \item Number of the $k$-point. For molecular (non-periodic)
    calculations only 1 $k$-point is printed.
  \item The folded matrix for the given $k$-point. It consists of
    \verb|NALLORB| lines $\times$ \verb|NALLORB| columns. If the
    matrix is not complex (\verb|REAL| is \verb|F|), every column
    contains two numbers (real and imaginary part).
  \end{itemize}
\end{itemize}

The files are produced if requested by \is{WriteHS} = \is{Yes} (see
section~\ref{WriteHS}).

\section{hamrealN.dat, overreal.dat}
\label{sec:hamreal}
\index{hamreal.dat}\index{overreal.dat} The files \verb|hamrealN.dat|
and \verb|overreal.dat| contain the real space Hamiltonian and overlap
matrices. The number \verb|N| in the filename \verb|hamrealN.dat|
indicates the spin channel. For spin unpolarised calculation it is 1,
for spin polarised calculation it is 1 and 2 for spin-up and
spin-down, respectively, while for non-colinear spin it is charge,
$x$, $y$ and $z$ for 1, 2, 3 and 4. Spin orbit is not currently
supported for this option.

Note: The sparse format contains only the "lower triangle" of the real
space matrix. For more details about the format and how to obtain the
upper triangle elements, see reference~\cite{dftbp-paper}. Also note,
that for periodic systems the sparse format is based on the
\emph{folded} coordinates of the atoms, resulting in translation
vectors (ICELL) which look surprising at first glance.

Only non-comment lines (lines not starting with "\#") are documented:
\begin{itemize}
\item Number of atoms in the system (\verb|NATOM|)
\item For every atom:
  \begin{itemize}
  \item Atom number (\verb|IATOM|), number of neighbours including the
    atom itself (\verb|NNEIGH|), number of orbitals on the atom
    (\verb|NORB|)
  \end{itemize}
\item For every neighbour of every atom:
  \begin{itemize}
  \item Atom number (\verb|IATOM1|), neighbour number (\verb|INEIGH|),
    corresponding image atom to the neighbour in the central cell
    (\verb|IATOM2F|), coefficients of the translation vector between
    the neighbour and its corresponding image (\verb|ICELL(1)|,
    \verb|ICELL(2)|, \verb|ICELL(3)|). Between the coordinates of the
    neighbour $\mathbf{r}_{\text{INEIGH}}$ and the image atom
    $\mathbf{r}_{\text{IATOM2F}}$ the relation
    \begin{equation*}
      \mathbf{r}_{\text{INEIGH}} = \mathbf{r}_{\text{IATOM2F}} + \sum_{i=1}^3
      \text{ICELL}(i)\, \mathbf{a}_i
    \end{equation*}
    holds, where $\mathbf{a}_i$ are the lattice vectors of the supercell.
  \item The corresponding part of the sparse matrix. The data block
    consists of \verb|NORB(IAT1)| lines and \verb|NORB(IAT2F)| columns.
  \end{itemize}
\end{itemize}

The files are produced if requested by \is{WriteRealHS} = \is{Yes}
(see section~\ref{WriteRealHS}).

\section{eigenvec.out, eigenvec.bin}
\label{sec:eigenvec}
\index{eigenvec.out}\index{eigenvec.bin}

These files contain the eigenvectors from the Hamiltonian, stored
either as plain text (eigenvec.out) or in the native binary format of
your system (eigenvec.bin).

The plain text format file \verb|eigenvec.out| contains a list of the values of
the components of each eigenvector for the basis functions of each atom. The
atom number in the geometry, its chemical type and the particular basis function
are listed, followed by the relevant value from the current eigenvector and then
the Mulliken population for that basis function for that level\index{state
  resolved Mulliken population}. The particular eigenvector, $k$-point and spin
channel are listed at the start of each set of eigenvector data. In the case of
non-colinear spin, the format is generalised for spinor wavefunctions. Complex
coefficients for both the up and down parts of the spinors are given (instead of
single eigenvector coefficient) followed by four values -- total charge, then
$(x,y,z)$ magnetisation.

The binary format file \verb|eigenvec.bin| contains the (unique) runId of the
DFTB+ simulation which produced the output followed by the values of the
eigenvectors. The eigenvector data is ordered so that the individual components
of the current eigenvector are stored, with subsequent eigenvectors for that
$k$-point following sequentially. All $k$-points for the current spin channel
are printed in this order, followed by the data for a second channel if spin
polarised.

The files are produced if requested by setting \is{WriteEigenvectors} =
\is{Yes}, with \is{EigenvectorsAsTxt} being also required to produce the plain
text file (see section~\ref{WriteEigenvectors} for details).

\section{charges.bin}
\label{sec:charges.bin}
\index{charges.bin}

The file charges.bin contains the orbitally-resolved charges for each atom,
ordered as the charges on each orbital of an atom for a given spin channel, then
each spin channel and finally over each atom. In later versions of \dftbp{} this
format includes a checksum for the total charge and magnetisation. In the case
of orbital potentials (p.~\pref{sec:DFTB+U}) the file also contains extra
population information for the occupation matrices.

This file is produced as part of the mechanism to restart SCC calculations, see
sections~\ref{RestartFrequency} and~\ref{MDRestartFrequency}.

\section{md.out}
\label{sec:md.out}
\index{md.out}

This file is only produced for \iscb{VelocityVerlet} calculations (See
p.~\pref{VelocityVerlet}). It contains a log of information generated during MD
calculations, and appended every \kw{MDRestartFrequency} steps. In the case of
small numbers of atoms and long MD simulations it may be useful to set
\is{WriteDetailedOut} to \is{No} and examine the information stored in this file
instead.

\section{Excited state results files}
\label{sec:tddftb_lr}
Several files are produced during excited state calculations depending on the
particular settings from section~\ref{ExcitedState}.

\textbf{Note:} in the case of degeneracies, the oscillator strengths depend on
arbitrary phase choices made by the ground state eigensolver. Only the sum over
the degenerate contributions is well defined for most single particle transition
properties, and label ordering of states may change if changing eigensolver or
platform. For the excited state, properties like the intensities for
individual excitations in degenerate manifolds again depend on phase choices
made by both the ground and excited eigensolvers.


\subsection{ARPACK.DAT}
\index{ARPACK.DAT}

Internal details of the ARPACK solution vectors, see the ARPACK
documentation~\cite{Lehoucq97arpackusers} for details.

\subsection{COEF.DAT}
\index{COEF.DAT}

Data on the projection of this specfic excited state onto the ground state
orbitals. For the specific exited state, the (complex) decomposition of its
single particle states onto the ground state single particle levels, together
with its fractional contribution to the full excited state are given.

General format:

\begin{tabular}{|l|l|}
\hline
{\small T F                                }&Legacy flags\\
{\small   1  1.9999926523  2.0000000000    }&level 1, fraction of total WF, 2.0\\
{\small-0.1944475716  0.0000000000 -0.1196876988  0.0000000000 ....    }&real then
imaginary projection of level 1\\
                                    &onto ground state 1, then ground state 2, etc.\\
{\small-0.1196876988  0.0000000000 -0.1944475703  0.0000000000 ....    }&\\
{\small.}&\\
{\small.}&\\
{\small.}&\\
{\small   2  1.9999866161  2.0000000000    }&level 2\\
{\small-0.2400145188  0.0000000000 -0.1767827333  0.0000000000 ....}& real then
imaginary projection of state 2\\
{\small.}&\\
{\small.}&\\
{\small.}&\\ \hline
\end{tabular}

\subsection{EXC.DAT}
\index{EXC.DAT}

Excitations data including the energies, oscilator strength, dominant Kohn-Sham
transitions and the symmetry.

Example first few transitions for C$_4$H$_4$:

\begin{verbatim}
     w [eV]       Osc.Str.         Transition         Weight      KS [eV]    Sym.

 =========================================

      5.551       0.5143882       11   ->    12        1.000       4.207      S
      5.592       0.0000000       10   ->    12        1.000       5.592      S
\end{verbatim}

Two examples of singlet transitions with energies of 5.551 and 5.592~eV. The
first is dipole allowed, the second not. In both cases they are transitions
primarilty (weight of 1.000) to single particle state 12, and are of singlet
character (``S'').

In the case of spin-polarised calculations, an additional column of values are
given instead of the symmetry, showing the level of spin contamination in the
state (labelled as \verb|D<S*S>|), with typically states where a magnitude of
less than 0.5 is usually considered reliable~\cite{garcia14Thesis}.

\subsection{SPX.DAT}
\index{SPX.DAT}

Single particle excitaions (SPX) for transitions beween filled and empty single
particle states of the ground state. These are given in increasing single
particle energy and show the oscillator strength and index of the Kohn-Sham-like
states that are involved.

\begin{verbatim}
       #      w [eV]       Osc.Str.        Transition

 ============================

       1      5.403       0.2337689       15   ->    16  
       2      5.403       0.2337689       14   ->    16  
       3      5.403       0.2337689       15   ->    17  
       4      5.403       0.2337689       14   ->    17  
       5      6.531       0.0000000       13   ->    16  
       6      6.531       0.0000000       12   ->    16  
\end{verbatim}

\subsection{TDP.DAT}
\index{TDP.DAT}

Detail of the magnitude and direction of the transition dipole from the ground
to excited states.

\subsection{TRA.DAT}
\index{TRA.DAT}

Decomposition of the transition from the ground state to the excited states. The
energy and spin symmetry are given together with the contributions from each of
the single particle transitions.

\subsection{TEST\_ARPACK.DAT}
\index{TEST\_ARPACK.DAT}

Tests on the quality of the eigenvalues and vectors returned by ARPACK. For the
$i^\mathrm{th}$ eigen-pair, the eigenvalue deviation corresponds to the
deviation from $\left( \langle \mathbf{x}_i | H | \mathbf{x}_i\rangle -
\epsilon_i \right)$, The eigen-vector deviation is a measure of rotation of the
vector under the action of the matrix: $\left| \left( H | \mathbf{x}_i\rangle -
\epsilon_i | \mathbf{x}_i\rangle \right) \right|_2$, the normalization deviation
is $\langle \mathbf{x}_i | \mathbf{x}_i\rangle - 1$ and finally largest failure
in orthogonality to other eigenvectors is given.

Example:\\
\begin{tabular}{lllll}
{\tt State} & {\tt Ei deviation} & {\tt Evec deviation} & {\tt Norm deviation} &
{\tt Max non-orthog}\\ {\tt 1} & {\tt -0.19428903E-15} & {\tt 0.80601119E-15} &
{\tt 0.19984014E-14} & {\tt 0.95562226E-15}\\ {\tt 2} & {\tt 0.27755576E-16} &
{\tt 0.85748374E-15} & {\tt 0.48849813E-14} & {\tt 0.36924443E-15}\\ {\tt 3} &
{\tt -0.12490009E-15} & {\tt 0.88607302E-15} & {\tt 0.88817842E-15} & {\tt
  0.60384195E-15}\\
\end{tabular}

\subsection{XCH.DAT}
\index{XCH.DAT}

Net charges on atoms in the specified excited state. The top line contains the
symmetry (Singlet or Triplet) and the number of the excited state. The next line
is the number of atoms in the structure followed by some header text. Then on
subsequent lines the number of each atom in the structure and its net charge are
printed.

\subsection{XplusY.DAT}
\index{XplusY.DAT}

Expert file with the RPA  $(X+Y)^{I\Sigma}_{ia}$ data for all the calculated
excited states.

Line 1: number of single particle excitations and the number of calculated
excited states\\
Line 2: Level number 1, nature of the state (S, T, U or D) then excitation
energy (in Hartree)\\
Line 3: expansion in the KS single particle transitions\\
.\\
.\\
.\\
Line 2: Level number 2, nature of the state (S, T, U or D) then excitation
energy (in Hartree)\\

\subsection{XREST.DAT}
\index{XREST.DAT}

Net dipole moment of the specified excited state in units of Debye.

%%% Local Variables: 
%%% mode: latex
%%% TeX-master: "manual"
%%% End: 
