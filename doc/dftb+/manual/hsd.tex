\chapter{The HSD format}
\label{sec:hsd}

The Human-friendly Structured Data (HSD) format is a structured input
format, which can be bijectively mapped onto a subset of the
XML-language. Its simplified structure and notation should make it a
more convenient user interface than reading and writing XML tags. This
section contains a brief overview of the most important aspects of
this format.

An input file in the HSD format consists basically of property
assignments of the form
\begin{verbatim}
  Property = value
\end{verbatim}
where the value \verb|value| was assigned to the property
\verb|Property|.  The value must be one of the following types
(detailed description of each follows later on):
\begin{itemize}
\item Scalar, such as
  \begin{itemize}
  \item integer
  \item real
  \item logical
  \item string
  \end{itemize}
\item list of scalars
\item method
\item list of further property assignments
\end{itemize}

An unquoted hash mark (\#) is interpreted as the start of a comment,
everything after it, up to the end of the current line, is ignored by
the parser (hash marks inside of quotes are taken as literals not
comments):
\begin{verbatim}
  # Entire line with comment
  Prop1 = "hell#oo"  # Note, that the first hashmark is quoted!
\end{verbatim}
The name of the properties, the methods and the logical values are case
insensitive, so the assignments
\begin{verbatim}
  Prop1 = 12
  prOP1 = 12
  Prop2 = Yes
  Prop2 = YES
\end{verbatim}
are pairwise identical. Quoted strings (specified either as a value
for a property or as a file name), however, are case sensitive.

Due to technical issues, the maximal line length is currently limited to 1024
characters. Lines longer than this are chopped without warning.

If a property, which should only appear once, is defined more than once, the
parser uses the \emph{first} definition and ignores all the other occurrences.
\emph{Thus specifying a property in the input a second time, does not override
  the first definition.} (For advanced use the HSD syntax also offers the
possibility of conditional overriding/extending of previous definitions. For
more details see \ref{sec:hsdext}.)


\section{Scalars and list of scalars}

The following examples demonstrate the assignments with scalar types:
\begin{verbatim}
  SomeInt = 1
  SomeInt2 = -3
  SomeRealFixedForm = 3.453
  SomeRealExpForm = 2.12e-45
  Logical1 = Yes
  Logical2 = no
  SomeString = "this is a string value"
\end{verbatim}

As showed above, real numbers can be entered in either fixed or
exponential form. The value for logical properties can be either
\verb|Yes| or \verb|No| (case insensitive). Strings should always be
enclosed in quotation marks, to make sure that they are treated as one
string and that they are not interpreted by the parser:
\begin{verbatim}
  String1 = "quoted string"
  String2 = this value is actually a list of 9 strings  # list of strings!
  String3 = "Method { ;"     # This is a string assignment
  String4 = Method {         # This is syntactically incorrect, since
                             # it tries to assign a method to String4
\end{verbatim}

A list of scalars is created by sequentially writing the scalars
separated by one or more spaces:
\begin{verbatim}
  PlottedLevels =  1 2 3
  Origin =  0.0 0.0 0.0
  ConfirmItTwice =  Yes Yes
  SpeciesNames =  "Ga" "As"
\end{verbatim}

The assignments statements are usually terminated by the end of the
line.  If the list of the assigned values goes over several lines, it
must be entered between curly (brace) brackets. In that case, instead
of the line end, the closing bracket will signal the end of the
assignment.  It is allowed to put a list of scalars in curly brackets,
even if it is only one line long.
\begin{verbatim}
  PlottedLevels = {
    1 2 3
  }
  Origin = { 0.0
  0.0 0.0 }
  Short = { 1 2 3 }
\end{verbatim}

If you want to put more than one assignment in a line, you have to
separate them with a semi-colon:
\begin{verbatim}
  Variable = 12; Variable2 = 3.0
\end{verbatim}

If a property should be defined as empty, either the empty list must be
assigned to it or it must be defined as an empty assignment
terminated by a semi-colon:
\begin{verbatim}
  EmptyProperty = {}
  EmptyProperty2 = ;
\end{verbatim}

Please note, that explicitly specifying a property to be empty is not
the same as not having specified it at all. In the latter case, the
parser substitutes the default value for that property (if there is a
default for it), while in the first case it interprets the property to
be empty. If a property without default value is not specified, the
parser stops with an appropriate error message.


\section{Methods and property lists}

Besides the scalar values and the list of scalars, the right hand side
of an assignment may also contain a method, which itself may contain
one or more scalar values or further property assignments as
parameters:
\begin{verbatim}
  Diagonaliser = LapackDAC {}     # Method without further params
  PlottedLevels = Range { 1 3 }   # Range needs two scalar params
  PlottedRegion = UnitCell {      # UnitCell needs a property list
    MinEdgeLength = 1.0           # as parameter
    SomeOtherProperty = Yes
  }
\end{verbatim}
The first assignment above is an example, where the method on the
right hand side does not need any parameters specified. Please note,
that even if no parameters are required, the opening and closing
brackets after the method are mandatory. If the brackets are missing,
the parser interprets the value as a string.

In the second assignment, the method \verb|Range| needs only two
integers as parameters, while for the method \verb|UnitCell| several
properties must be specified. A method may contain either nothing or
scalars or property assignments, but never scalars and property
assignments together. So the following assignment would be invalid:
\begin{verbatim}
  InvalidSpecif = SomeMethod {
    1 2 3
    Property1 = 12
    "Some strings here"
  }
\end{verbatim}

Very often a value for the property is represented by a list of
further property assignments (as above, but without naming an
explicit method beforehand). In that case, the property assignments must
be put between curly brackets (property list):
\begin{verbatim}
  Options = {
    SubOption1 = 12
    Suboption2 = "string"
  }
\end{verbatim}


\section{Modifiers}

Each property may carry a modifier, which changes the interpretation
of the assigned value:
\begin{verbatim}
  LatticeConstant [Angstrom] = 12.23
\end{verbatim}
Here, the property \verb|LatticeConstant| possesses the \verb|Angstrom|
modifier, so the specified value will be interpreted to be in
\AA{}ngstr\"om instead of the default length unit. Specifying a modifier
for a property which is not allowed to carry one leads to parsing error.

The syntax of the HSD format also allows methods (used as values on the right
hand side of an assignment) to carry modifiers, but this is usually not used in
the current input structures.

Sometimes, the assigned value to a property contains several values
with different units, so that more than one modifiers can be
specified. In that case, the modifiers must be separated by a comma.
\begin{verbatim}
  VolumeAndChargePerElement [Angstrom^3,au] = {
     1.2   0.3   # first element
     4.2   0.1   # second element
  }
\end{verbatim}
You have to specify either no modifier or all modifiers. If you want
specify the default units for some of the quantities, you can omit the
name of the appropriate modifier, but you must include the separating
comma:
\begin{verbatim}
  # Specifying the default unit for the charge. Note the separating comma!
  VolumeAndChargePerElement [Angstrom^3,] = {
     1.2   0.3   # first element
     4.2   0.1   # second element
  }
\end{verbatim}
Specifying not enough or too many modifiers leads to parser error.

\section{File inclusion}

It is possible to include files in an HSD-formatted input by using the
\verb|<<<| and \verb|<<+| operators. The former includes the specified
file as raw text without parsing it, while latter parses the included
text:
\begin{verbatim}
  Geometry = GenFormat {
    <<< "geo_start.gen"
  }
  Basis = {
    <<+ "File_containing_the_property_definitons_for_the_basis"
  }
\end{verbatim}
The file included with the \verb|<<+| operator must be a valid HSD
document in itself.


\section{Processing}

After having parsed and processed the input file, the parser writes
out the processed input to a separate file in HSD format. This file
contains the internal representation for all properties, which can be
specified by the user. In particular, all default values are explicitly
set and all automatic definitions (e.g. ranges) are converted to their
internal representations.

Assuming the following example as input
\begin{verbatim}
  # Lattice contant specified in Angstrom.
  # Internal representation uses Bohr, so it will be converted.
  LatticeConstant [Angstrom] = 12.0

  # This property is not set, as its commented out, so the
  # default value will be set for this (let's assume, it's Yes)
  #DoAProperJob = No

  # Plotted levels specified as a range with parameter 1:3.
  # This will be replaced by an explicit listing of the levels
  PlottedLevels = { 1:3 }
\end{verbatim}
the parsed and processed input (written to a special file) should look
something like
\begin{verbatim}
  LatticeConstant = 22.676713499923075
  DoAProperJob = Yes
  PlottedLevels = {
    1 2 3
  }
\end{verbatim}
If you want to reproduce your calculation later, you should use this
processed input. It should give you identical results, even if the
default setting for some properties had been changed in the code.

\section{Extended format}
\label{sec:hsdext}

As stated earlier, if a property, which should be defined only once,
occurs more than once in the input, the parser uses per default the
first definition and ignores all the others. Sometimes this is not the
desired behaviour, therefore, the HSD format also offers the
possibility to override properties that were set earlier.  This
feature can be very useful for scripts which are generate HSD input
based on some user provided template.  By just appending a few lines
to the end of the user provided input the scripts can make sure that
certain properties are set correctly. Thus, the script can modify the
user input, without having to parse it at all.

The parser builds internally an XML DOM-tree from the HSD input. For
every property or method name an XML tag with the same name (but
lowercased) is created, which will contain the value of the property
or the method. If the value contains further properties or methods,
new XML tags are created inside the original one. Shortly, the HSD
input is mapped on a tree, whereas the assignment and the containment
(equal sign and curly brace) are turned into a parent-child
relationships.\footnote{In the internal tree representation of the HSD
  input there is no difference between properties and methods, both
  are just elements capable to contain some value or further
  elements. The differentiation in the HSD input is artificial and is
  only for human readability (equal sign after property names, curly
  brace after method names),} As an example an HSD input and the
corresponding XML-representation is given below:

\begin{minipage}{1.0\linewidth}
  \begin{center}
    \begin{minipage}[t]{5.5cm}
\begin{verbatim}
  Level0Elem1 = 1
  Level0Elem2 = { 1 2 3 }
  Level0Elem3 = {
    Level1Elem1 = 12
    Level1Elem2 = Level2Elem1 {

      Level3Elem1 = "abcd"
      Level3Elem2 = {
        Level4Elem1 = 12
      }
    }
  }
\end{verbatim}
    \end{minipage}
    \hspace*{1cm}
    \begin{minipage}[t]{7.5cm}
\begin{verbatim}
  <level0elem1>1</level0elem1>
  <level0elem2>1 2 3</level0elem2>
  <level0elem3>
     <level1elem1>12</level1elem1>
     <level1elem2>
        <level2elem1>
           <level3elem1>"abcd"</level3elem1>
           <level3elem2>
              <level4elem1>12</level4elem1>
           </level3elem2>
        </level2elem1>
     </level1elem2>
  </level0elem3>
\end{verbatim}
    \end{minipage}
  \end{center}
\end{minipage}

By prefixing property and method names, the default behaviour of the parser can
be overridden. Instead of creating a new tag (on the current encapsulation level)
with the appropriate name, it will look for the \emph{first occurrence} of the
given tag and will process that one. Depending of the prefix character, the tag
is processed in the following ways:
\begin{description}
\item[+:] If the tag exists already, it's value is modified, otherwise the
  parser stops.
\item[?:] If the tag exists already, it's value is modified, otherwise the
  parser ignores the prefixed HSD construct.
\item[*:] If the tag exists already, it's value is modified, otherwise it is
  created (and then it's value is modified).
\item[/:] If the tag does not exist yet, it is created and modified, otherwise
  the prefixed HSD construct is ignored.
\item[!:] The tag is newly created and modified. If it exists already, the old
  occurrence is deleted first.
\end{description}

The way the value of the tag is going to be modified, is ruled by the constructs
inside the prefixed property or method name. If the parser finds non prefixed
constructs here, the appropriate tags are just added, otherwise the behaviour is
determined by the rules above, just acting one level deeper in the tree. The
following examples should make this a little bit more clear.

\begin{itemize}
\item Changing the value of \is{Level0Elem1} to \is{3}. If the element does not
  exist, it should be created with the value \is{3}.
\begin{verbatim}
  !Level0Elem1 = 3
\end{verbatim}

\item Changing the value of \is{Level0Elem3/Level1Elem1} to \is{21} (the slash
  indicates the parent-child relationship). If the element does not exist, stop
  with an error message:
\begin{verbatim}
  # Make sure the containing element exists. If yes, go inside, otherwise die.
  +Level0Elem3 = {
    # Set the value of Level1Elem1 or die, if it does not exist.
    +Level1Elem1 = 21
  }
\end{verbatim}
  Please note, that each tag in the path must be prefixed. Using the following
  construct instead of the original one
\begin{verbatim}
  # Not prefixed, so it creates a new tag with empty value
  Level0Elem3 = {
    # The new tag doesn't contain anything, so the parser stops here
    +Level1Elem1 = 21
  }
\end{verbatim}
  would end with an error message. Since \is{Level0Elem1} is not prefixed here,
  a tag is created for it with an empty value (no children). It does not matter,
  whether the tag already existed before or not, a new tag is created and
  appended as the last element (last child) to the current block. Then the
  parser is processing its value. Due to the \is{+Level1Elem1} directive it is
  looking for a child tag \is{<level1elem1>}. Since the tag was newly created,
  it does not contain any children, so the parser stops with an error message.

\item Create a new tag \is{Level1Elem3} inside \is{Level0Elem3} with some
  special value. If the tag already exists, replace it.
\begin{verbatim}
  # Modifying the children of Level0Elem3 or dying if not present
  +Level0Elem3 = {
    # Replacing or if not existent creating Level1Elem3
    !Level1Elem3 = NewBlock {
       NewValue1 = 12
    }
\end{verbatim}
  This example also shows, that the value for the new property can be any
  arbitrary complex HSD construct.

\item Provide a default value \is{"string"} for
  \is{Level0Elem3/Level1Elem2/Level2Elem1/Level3Elem1}. If the tag is already
  present do not change its value.
\begin{verbatim}
  # Modify Level0Elem3 or create it if non-existent
  *Level0Elem3 = {
  # Modify Level1Elem2 and Level2Elem1 or create them if non-existent
    *Level1Elem2 = *Level2Elem1 {
      # Create Level3Elem1 if non-existent with special value.
      /Level3Elem1 = "string"
    }
  }
\end{verbatim}

\item If \is{Level0Elem3/Level1Elem2} has the value \is{Level2Elem1}, make sure
  that \is{Level3Elem1} in it exists, and has \is{""} as value. If
  \is{Level1Elem2} has a different value, do not change anything.
\begin{verbatim}
  # If Level0Elem3 is present, process it, otherwise skip this block
  ?Level0Elem3 = {
    # The same for the next two containers
    ?Level1Elem2 = ?Level2Elem1 {
      # Create or replace Level3Elem1
      !Level3Elem1 = ""
    }
  }
\end{verbatim}


\section{Index selection expressions}
\label{sec:index_selection}

When specific elements should be selected from a list, the HSD parser allows for the usage of
special index selection expressions.  Typical examples are the selection of atoms (as in the
\is{MovedAtoms} or \is{Atoms} keywords) or the selection of energy levels, k-points and spins (as in
the \waveplot{} keywords \is{PlottedKPoints}, \is{PlottedSpins}, \is{PlottedLevels}).

The expressions might contain the logical operators ``not'' (``\verb|!|''), ``and'' (``\verb|&|'')
and ``or'' (`` '', i.e.\ white space). The ``not'' operator has the highest and the ``or'' operator
the lowest precedence. Parentheses can be used for grouping together operations. If atoms are being
selected, atom type names can be used to select all atoms of a given type. The first element of the
list has the index one. Negative indices are counted backwards from the last element of the list,
with -1 corresponding to the last element. Index ranges can be specified as \verb|start:end| (i.e.\
without white space as one word!), which inclusively selects all elements between \verb|start| and
\verb|end|.

Below you find a few examples for selection expressions:

\begin{verbatim}
# Atoms 1, 2, 3, 4
MovedAtoms = 1 2 3 4 5

# Atoms within the range [1, 4] -> 1, 2, 3, 4
MovedAtoms = 1:4

# All atoms (from first to last)
MovedAtoms = 1:-1

# All atoms up to (and including) the fourth from the end
# (= all atoms apart the last 3)
MovedAtoms = 1:-4

# All Si atoms and atoms 2 and 4.
MovedAtoms = Si 2 4

# Any atoms from 2 to 6 which are Si atoms
MovedAtoms = 2:6 & Si

# All atoms, apart from atoms 2, 3, 4, 5, 6
MovedAtoms = !2:6

# All atoms from 2 to 6 which are not Si atoms
MovedAtoms = 2:6 & !Si

# All atoms apart from 1:3 and 5:7
MovedAtoms = !(1:3 5:7)

# All non-Si atoms in the ranges 1 to 10, 21 to 30 and 41 to 50.
MovedAtoms = !Si & (1:10 21:30 41:50)
\end{verbatim}

The syntax of the index selection can be described by following formal grammar:

\begin{verbatim}
expr := addTerm { addTerm }
addTerm := mulTerm { "&" mulTerm }
mulTerm := term |  "!"term
term := selector | "("expr")"
selector := index{":"index} | speciesName
index := {-}[0-9]+
speciesName := [a-zA-Z][0-9a-zA-Z_]*
\end{verbatim}

\end{itemize}
