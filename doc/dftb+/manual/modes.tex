\chapter{\modes{}}

The \modes{} program calculates vibrational modes using data created by
\dftbp{}.


\section{Input for \modes}

The input file for \modes{} must be named \verb|modes_in.hsd| and should be a
Human-friendly Structured Data (HSD) formatted file (see Appendix
\ref{sec:hsd}). The program can read the input in XML instead of HSD format if
the input file is \verb|modes_in.xml|. The input file must be present in the
working directory. As with {\dftbp} to prevent ambiguity, the parser refuses to
read in any input if two types of input file are present.

The table below contains the list of the properties, which must occur in the
input file \verb|modes_in.hsd|:

\begin{ptableh}
  \kw{Geometry} & p|m &  & - & \pref{sec:dftbp.Geometry} \\
  \kw{Hessian} & p & & \cb & \pref{sec:modes.Hessian} \\
  \kw{SlaterKosterFiles} &p|m&  & - &  \\
\end{ptableh}

Additionally optional definitions may be present:
\begin{ptableh}  
  \kw{DisplayModes} & p & & - & \pref{sec:modes.DisplayModes} \\
  \kw{Atoms} & i+|m &  & 1:-1 & \\
\end{ptableh}

\begin{description}
\item[\is{Geometry}] Specifies the geometry for the system to be
  calculated.  See p.~\pref{sec:dftbp.Geometry}.
\item[\is{Hessian}] Contains the second derivatives matrix of the
  system energy with respect to atomic positions. See
  p.~\pref{sec:modes.Hessian}.
\item[\is{SlaterKosterFiles}] Name of the Slater-Koster files for every atom
  type pair combination. See p.~\pref{sec:dftbp.SlaterKosterFiles}.
\item[\is{DisplayModes}] Optional settings to plot the eigenmodes of the
  vibrations. See p.~\pref{sec:modes.DisplayModes}.
\item[\is{Atoms}] Optional list of atoms, ranges of atoms and/or the species of
  atoms for which the Hessian has been supplied. \emph{This must be equivalent
    to the setting you used for \is{MovedAtoms} in your \dftbp{} input when
    generating the Hessian.}
\end{description}


\subsection{Hessian\{\}}
\label{sec:modes.Hessian}

Contains the second derivatives\index{Hessian} of the energy supplied by
{\dftbp}, see p.~\pref{sec:dftbp.SecondDerivatives} for details of the options
to generate this data. The derivatives matrix must be stored as the following
order: For the $i$, $j$ and $k$ directions of atoms $1 \ldots n$
as $$\frac{\partial^2 E}{\partial x_{i1} \partial x_{i1}} \frac{\partial^2
  E}{\partial x_{j1} \partial x_{i1}} \frac{\partial^2 E}{\partial x_{k1}
  \partial x_{i1}} \frac{\partial^2 E}{\partial x_{i2} \partial x_{i1}}
\frac{\partial^2 E}{\partial x_{j2} \partial x_{i1}} \frac{\partial^2
  E}{\partial x_{k2} \partial x_{i1}} \ldots \frac{\partial^2 E}{\partial x_{kn}
  \partial x_{kn}}$$

{\em Note}: for supercell calculations, the modes are currently
obtained at the $\mathbf{q}=0$ point, irrespective of the k-point
sampling used.


\subsection{DisplayModes\{\}}
\label{sec:modes.DisplayModes}

Allows the eigenvectors of the system to be plotted out if present

\begin{ptable}
\kw{PlotModes} & i+|m &  & 1:-1 & \\
\kw{Animate} & l & & Yes &  \\
\kw{XMakeMol} & l & & Yes &  \\
\end{ptable}
\begin{description}
\item[\is{PlotModes}] Specifies list of which eigenmodes should be
  plotted as xyz files. Remember that there are $3N$ modes for the
  system (including translation and rotation).
\item[\is{Animate}] Produce separate animation files for each mode or
  a single file multiple modes where the mode vectors are marked for
  each atom.
\item[\is{XMakeMol}] Adapt xyz format output for XMakeMol dialect xyz
  files.
\end{description}
