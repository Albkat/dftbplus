\chapter{\modes{} program}
\label{app:modes}

The \modes{} program calculates vibrational modes using data created by
\dftbp{}.


\section{Input for \modes}

The input file for \modes{} must be named \verb|modes_in.hsd| and should be a
Human-friendly Structured Data (HSD) formatted file (see Appendix \ref{sec:hsd})
and must be present in the working directory.

The table below contains the list of the properties, which must occur in the
input file \verb|modes_in.hsd|:

\begin{ptableh}
  \kw{Geometry} & p|m &  & - & \pref{sec:dftbp.Geometry} \\
  \kw{Hessian} & p & & \cb & \pref{sec:modes.Hessian} \\
\end{ptableh}

Additionally optional definitions may be present:
\begin{ptableh}
  \kw{DisplayModes} & p & & - & \pref{sec:modes.DisplayModes} \\
  \kw{Atoms} & i+|m &  & 1:-1 & \\
  \kw{WriteHSDInput} & l & & No & \\
  \kw{RemoveTranslation} & l & & No & \\
  \kw{RemoveRotation} & l & & No & \\
  \kw{SlaterKosterFiles} &p|m&  & - &  \\
  \kw{Masses} & p & & & \pref{sec:modes.Masses} \\
  \kw{BornCharges} & p & & \cb & \pref{sec:modes.BornCharges} \\
  \kw{BornDerivs} & p & & \cb & \pref{sec:modes.BornDerivs} \\
\end{ptableh}

\begin{description}
\item[\is{Geometry}] Specifies the geometry for the system to be
  calculated.  See p.~\pref{sec:dftbp.Geometry}.
\item[\is{Hessian}] Contains the second derivatives matrix of the
  system energy with respect to atomic positions. See
  p.~\pref{sec:modes.Hessian}.
\item[\is{SlaterKosterFiles}] Name of the Slater-Koster files for
  every atom type pair combination. See
  p.~\pref{sec:dftbp.SlaterKosterFiles}. This is used to obtain the
  masses, so if these are explicitly set using \kwcb{Masses}, it is
  not required.
\item[\is{DisplayModes}] Optional settings to plot the eigenmodes of the
  vibrations. See p.~\pref{sec:modes.DisplayModes}.
\item[\is{Atoms}] Optional list of atoms, ranges of atoms and/or the species of
  atoms for which the Hessian has been supplied. \emph{This must be equivalent
    to the setting you used for \is{MovedAtoms} in your \dftbp{} input when
    generating the Hessian.}
\item[\is{WriteHSDInput}] Specifies, if the processed input should be written
  out in HSD format. (You shouldn't turn it off without good reason.)
\item[\is{RemoveTranslation}] Explicitly set the 3 translational modes of the
  system to be at 0 frequency.
\item[\is{RemoveRotation}] Explicitly set the rotation modes of the system to be
  at 0 frequency. Note, for periodic systems, this is usually incorrect (if used
  for a molecule full inside the central cell, it may be harmless).
\item[\is{Masses}] If present, replace the atomic masses from the Slater-Koster files. See
  p.~\pref{sec:modes.Masses}
\end{description}


\subsection{Hessian\{\}}
\label{sec:modes.Hessian}

Contains the second derivatives\index{Hessian} of the total energy,
see p.~\pref{sec:dftbp.SecondDerivatives} for details of the \dftbp{}
options to generate this data. The derivatives matrix must be stored
as the following order: For the $i$, $j$ and $k$ directions of atoms
$1 \ldots n$ as
\begin{equation*}
  \frac{\partial^2 E}{\partial x_{i1} \partial x_{i1}} \frac{\partial^2
    E}{\partial x_{j1} \partial x_{i1}} \frac{\partial^2 E}{\partial x_{k1}
    \partial x_{i1}} \frac{\partial^2 E}{\partial x_{i2} \partial x_{i1}}
  \frac{\partial^2 E}{\partial x_{j2} \partial x_{i1}} \frac{\partial^2
    E}{\partial x_{k2} \partial x_{i1}} \ldots \frac{\partial^2 E}{\partial
    x_{kn} \partial x_{kn}}
\end{equation*}

{\em Note}: for supercell calculations, the modes are currently
obtained at the $\mathbf{q}=0$ point, irrespective of the k-point
sampling used.

\subsection{BornCharges\{\}}
\label{sec:modes.BornCharges}
\index{Infrared vibrational intensities}

If the mixed second derivatives of the energy with respect to electric
field and position are available ($Z^\star$ values), these can be used
to gain an estimate of the infrared activitiy of the vibrational
modes. The resulting transition strengths (in arbitrary units) are
then printed. The derivatives are also equivalent to the first
derivatives of the dipole moment with respect to atomic positions, or
the derivatives of forces with respect to external electric field.

The Born charges can be generated by \dftbp{} by evaluating finite
difference derivatives (see sec.~\ref{sec:dftbp.SecondDerivatives}),
at the same time that the hessian matrix is calculated. The resulting
Born charges are stored in the file {\it born.out}.

The Born charge data is ordered in the same way as the hessian
information:
\begin{align*}
  \frac{\partial \mu_x}{\partial x_{i1}} \frac{\partial \mu_y}{\partial x_{i1}}
  \frac{\partial \mu_z}{\partial x_{i1}}\\
  \frac{\partial \mu_x}{\partial x_{j1}} \frac{\partial \mu_y}{\partial x_{j1}}
  \frac{\partial \mu_z}{\partial x_{j1}}\\
  \frac{\partial \mu_x}{\partial x_{k1}} \frac{\partial \mu_y}{\partial x_{k1}}
  \frac{\partial \mu_z}{\partial x_{k1}}\\
  .\\
  .\\
  .\\
  \frac{\partial \mu_x}{\partial x_{kn}} \frac{\partial \mu_y}{\partial x_{kn}}
  \frac{\partial \mu_z}{\partial x_{kn}}\\
\end{align*}

The input to read the Born charges is:
\begin{verbatim}
  BornCharges = {
    <<< born.out
  }
\end{verbatim}

\subsection{BornDerivs\{\}}
\label{sec:modes.BornDerivs}
\index{Raman intensities}

If the mixed third derivatives of the energy with respect to electric
field and position are available ($Z^\star$ values differentiated with
respect to applied field), these can be used to gain an estimate of
the Raman activity of the vibrational modes. As implemented, this is
the change in the polarisability magnitude along the vibrational
modes' directions, not the Raman intensity (see for example
\cite{Ruud2009} for a full discussion of calculating Raman
intensities).

The resulting transition strengths (in atomic units) are then printed.

The input to read the Born charge derivatives is:
\begin{verbatim}
  BornDerivs = {
    <<< bornderiv.out
  }
\end{verbatim}

These derivatives can be evaluated with \dftbp{} by calculating the
second derivatives of the energy (see
\ref{sec:dftbp.SecondDerivatives}) while also requesting either the
static or dynamic electric polarisability (see
section~\ref{sec:dftbp.eperturb}). The resulting Born charges
derivatives are stored in the file {\it bornderiv.out}.

\subsection{DisplayModes\{\}}
\label{sec:modes.DisplayModes}

Allows the eigenvectors of the system to be plotted out if present

\begin{ptable}
\kw{PlotModes} & i+|m &  & 1:-1 & \\
\kw{Animate} & l & & Yes &  \\
\end{ptable}
\begin{description}
\item[\is{PlotModes}] Specifies list of which eigenmodes should be
  plotted as xyz files. Remember that there are $3N$ modes for the
  system (including translation and rotation).
\item[\is{Animate}] Produce separate animation files for each mode or
  a single file multiple modes where the mode vectors are marked for
  each atom.
\end{description}

\subsubsection{Masses}
\label{sec:modes.Masses}

Provides values of atomic masses for specified atoms, ranges of atoms or chemical species. This is
useful for example to set isotopes for specific atoms in the system.

\begin{ptable}
  \kw{Mass} & p & & & \\
\end{ptable}

Any atoms not given specified masses will use the default values from the appropriate homonuclear
Slater-Koster file. An example is given below:
\begin{verbatim}
  Masses {
    Mass {
      Atoms = H
      MassPerAtom [amu] = 1.007825
    }
    Mass {
      Atoms = C
      MassPerAtom [amu] = 13.003355
    }
    Mass {
      Atoms = 1:2
      MassPerAtom [amu] = 2.014102
    }

  }
\end{verbatim}
where \kw{Atoms} specifies the atom or atoms which each have a mass of \kw{MassPerAtom} assigned.
