
\newcommand{\dftbp}{\textsc{DFTB+}} %% noodle (alias dftb+)
\newcommand{\dftb}{\textsc{DFTB}}             %% dftb
\newcommand{\modes}{\textsc{modes}}
\newcommand{\waveplot}{\textsc{Waveplot}}
\newcommand{\setupgeom}{\textsc{setupgeom}}
\newcommand{\dptools}{\textsc{dp\_tools}}

\newcommand{\pref}[1]{\pageref{#1}}           %% pageref
\newcommand{\cb}{\{\}}                        %% curly braces
\newcommand{\is}[1]{\textsf{#1}}              %% input style
\newcommand{\iscb}[1]{\is{#1\cb}}              %% input style
\newcommand{\isl}[2]{\hyperref[#2]{\textsf{#1}}}  %% input style
\newcommand{\islcb}[2]{\hyperref[#2]{\iscb{#1}}} %% input style
\newcommand{\kw}[1]{\is{#1}\index{#1@\is{#1}}}   %% keyword (with index entry)
\newcommand{\kwl}[2]{\is{#1}\index{#1@\is{#1}}\label{#2}} %% keyword (with
%% index entry and label)
\newcommand{\kwcb}[1]{\iscb{#1}\index{#1\cb@\iscb{#1}}} %% kword with curly braces

%%% Sections, subsections etc. as hyperreference targets.
\newcommand{\htchapter}[1]{\chapter{\kw{#1}}\label{#1}}
\newcommand{\htsection}[1]{\section{\kw{#1}}\label{#1}}
\newcommand{\htsubsection}[1]{\subsection{\kw{#1}}\label{#1}}
\newcommand{\htsubsubsection}[1]{\subsubsection{\kw{#1}}\label{#1}}
\newcommand{\htparagraph}[1]{\paragraph{\kw{#1}}\label{#1}}
\newcommand{\htsubparagraph}[1]{\subparagraph{\kw{#1}}\label{#1}}
\newcommand{\modif}[1]{\is{[#1]}}
\newcommand{\modtype}[1]{{\textrm{\textit{#1}}}}
%%\newcommand{\modtype}[1]{\textsf{[}\textit{#1}\textsf{]}\ }
%%\newcommand{\modexpl}[1]{\textsf{[#1]}\ }


%%% Sections, subsections etc. as hyperreference targets (if title is a 
%%% curly braced keyword).
\newcommand{\htcbchapter}[1]{\chapter{\kwcb{#1}}\label{#1}}
\newcommand{\htcbsection}[1]{\section{\kwcb{#1}}\label{#1}}
\newcommand{\htcbsubsection}[1]{\subsection{\kwcb{#1}}\label{#1}}
\newcommand{\htcbsubsubsection}[1]{\subsubsection{\kwcb{#1}}\label{#1}}
\newcommand{\htcbparagraph}[1]{\paragraph{\kwcb{#1}}\label{#1}}
\newcommand{\htcbsubparagraph}[1]{\subparagraph{\kwcb{#1}}\label{#1}}

%%% Table of properties
\renewcommand{\tabularxcolumn}[1]{>{\raggedright\arraybackslash}p{#1}}
\newenvironment{ptable}{
  \par
  \begin{minipage}{\linewidth}
    \begin{tabular*}{\linewidth}{|>{\sf}lc>{\sf}l@{\extracolsep{\fill}}>{\sf}lr|}
      \hline
    }
    {
      \hline
    \end{tabular*}
  \end{minipage}
}



\newenvironment{ptableh}{
  \begin{ptable}
    \textrm{Name} & \textrm{Type} & \textrm{Condition} &
    \textrm{Default} & \textrm{Page} \\
    \hline }
  {
  \end{ptable}
}


\newenvironment{unittable}[1]{
  \par
  \begin{minipage}{\linewidth}
    \begin{tabular*}{\linewidth}{l@{\extracolsep{\fill}}l}
      \multicolumn{2}{l}{\textbf{#1:}}\\
    }
    {
    \end{tabular*}
  \end{minipage}
}


\addtolength{\hoffset}{-1.0cm}
\addtolength{\textwidth}{2.0cm}
\addtolength{\voffset}{-1.0cm}
\addtolength{\textheight}{1.5cm}

\renewcommand{\ttdefault}{\sfdefault}

%% Inverse parskip
\newcommand{\invparskip}{\vspace*{-\parskip}}

