
\chapter{Description of the gen format}
\label{app:gen}

The general (gen) format can be used to describe clusters and
supercells. It is based on the xyz format introduced with xmol, and
extended to periodic structures. Unlike some earlier implementations
of gen, the format should not include any neighbour mapping
information.

The first line of the file contains the number of atoms, $n$, followed
by the type of geometry. \is{C} for cluster (non-periodic), \is{S} for
supercell in Cartesian coordinates or \is{F} for supercell in
fractions of the lattice vectors. The supercells are periodic in 3
dimensions.

The second line contains the chemical symbols of the elements present
separated by one or more spaces.  The following $n$ lines contain a
list of the atoms. The first number is the atom number in the
structure (this is currently ignored by the program). The second
number is the chemical type from the list of symbols on line 2. Then
follow the coordinates. For \is{S} and \is{C} format, these are $x$,
$y$, $z$ in {\AA}, but for \is{F} they are fractions of the three
lattice vectors.

If the structure is a supercell, the next line after the atomic
coordinates contains the coordinate origin in {\AA} (this is ignored
by the parser). The last three lines are the supercell vectors in
{\AA}. These four lines are not present for clusters.

Example: Geometry of GaAs with 2 atoms in the fractional supercell
format
\begin{verbatim}
  2  F
  Ga As
  1 1     0.0  0.0  0.0
  2 2     0.25 0.25 0.25
  0.000000     0.000000     0.000000
  2.713546     2.713546     0.      
  0.           2.713546     2.713546
  2.713546     0.           2.713546
\end{verbatim}
